\documentclass[conference]{IEEEtran} 

%2019 hl ref: https://en.wikibooks.org/wiki/LaTeX/Mathematics#Matrices_and_arrays
\ifCLASSINFOpdf
  % \usepackage[pdftex]{graphicx}
  % declare the path(s) where your graphic files are
  % \graphicspath{{../pdf/}{../jpeg/}}
  % and their extensions so you won't have to specify these with
  % every instance of \includegraphics
  % \DeclareGraphicsExtensions{.pdf,.jpeg,.png}
\else
  % or other class option (dvipsone, dvipdf, if not using dvips). graphicx
  % will default to the driver specified in the system graphics.cfg if no
  % driver is specified.
  % \usepackage[dvips]{graphicx}
  % declare the path(s) where your graphic files are
  % \graphicspath{{../eps/}}
  % and their extensions so you won't have to specify these with
  % every instance of \includegraphics
  % \DeclareGraphicsExtensions{.eps}
\fi
% graphicx was written by David Carlisle and Sebastian Rahtz. It is
% required if you want graphics, photos, etc. graphicx.sty is already
% installed on most LaTeX systems. The latest version and documentation
% can be obtained at: 
% http://www.ctan.org/pkg/graphicx
% Another good source of documentation is "Using Imported Graphics in
% LaTeX2e" by Keith Reckdahl which can be found at:
% http://www.ctan.org/pkg/epslatex
%
% latex, and pdflatex in dvi mode, support graphics in encapsulated
% postscript (.eps) format. pdflatex in pdf mode supports graphics
% in .pdf, .jpeg, .png and .mps (metapost) formats. Users should ensure
% that all non-photo figures use a vector format (.eps, .pdf, .mps) and
% not a bitmapped formats (.jpeg, .png). The IEEE frowns on bitmapped formats
% which can result in "jaggedy"/blurry rendering of lines and letters as
% well as large increases in file sizes.
%
% You can find documentation about the pdfTeX application at:
% http://www.tug.org/applications/pdftex

% correct bad hyphenation here
\hyphenation{op-tical net-works semi-conduc-tor}


%\usepackage{amsmath} 
%HL 2018-1-21: not sure if IEEE accepts this

\usepackage{graphicx}
\graphicspath{images/}
%HL 2018-1-21 add this image folder


\begin{document}

\title{Lecture Note 1 on Gradient Descent}

% author names and affiliations
% use a multiple column layout for up to three different
% affiliations
\author{
\IEEEauthorblockN{Harry Li, Ph.D.}
\IEEEauthorblockA{Department of Computer Engineering, College of Engineering\\
San Jose State University, San Jose, CA 95192, USA\\
Email$^{\dagger}$: harry.li@ctione.com\\}
}

% make the title area
\maketitle

% As a general rule, do not put math, special symbols or citations
% in the abstract
\begin{abstract}
In this lecture note, we give a gradient descent
example for its applications 
in Neural Networks (NN), e.g., the concept of 
the negative gradient -$\nabla f$ follows
the direction of steepest descent. 
\end{abstract}

% no keywords

% For peer review papers, you can put extra information on the cover
% page as needed:
% \ifCLASSOPTIONpeerreview
% \begin{center} \bfseries EDICS Category: 3-BBND \end{center}
% \fi
%
% For peerreview papers, this IEEEtran command inserts a page break and
% creates the second title. It will be ignored for other modes.
\IEEEpeerreviewmaketitle

\section{Introduction}
%Section 1
% no \IEEEPARstart 
 
In this lecture note, we give a gradient descent
example for Neural Networks (NN) applications. 
In particular, the basic concept of the 
negative gradient -$\nabla f$ follows
the direction of steepest descent of a given 
function $f$ which can be an error function.  
% You must have at least 2 lines in the paragraph with the drop letter
% (should never be an issue) 

%\hfill mds
 
%\hfill August 26, 2015

\section{Partial Derivative vs. Gradient}
%Section 2

Given a scalar-valued multivariable functions,
e.g. the function with a multidimensional input
$x_1, x_2, ..., x_n$, and 
a one-dimensional output as 
$y=f(x_1, x_2, ..., x_)$, where 
$f : R^n \to R$. The partial 
derivative of $f(x_1, x_2, ..., x_n)$ with respect
to $x_i$ for $i=1,2,...,n$: 

\begin{equation}
\frac{ \partial f } { \partial x_i } = 
\lim_{\delta x\to 0} 
\frac{ f(x_1,.., x_i + \delta x_i, .., x_n) 
- f(x_1,..., x_i, ..., x_n)} 
{ \delta x_i }
\end{equation}
The gradient of the function $\nabla f$,
is the collection of all its partial derivatives into 
a vector form, e.g., vector valued function [1]. 

\begin{equation}
\nabla f =  
\left(\begin{array}{c} 
\frac{ \partial f} { \partial x_1 } \\ 
\frac{ \partial f} { \partial x_2 } \\
... \\
\frac{ \partial f} { \partial x_i } \\
... \\
\frac{ \partial f} { \partial x_n } 
\end{array}\right)
\end{equation}
Suppose standing on a surface of 
$f(x_1, x_2, ..., x_n)$ at a point 
$(x_1, x_2, ..., x_n)$, 
$\nabla f$ tells you which direction to 
travel to increase the value of $f$ most rapidly.
Hence, we make the following claim. 

\textbf{Claim 1.} 
\textit{The negative gradient -$\nabla f$ follows
the direction of steepest descent [2].}

\section{Gradient Example}

Example 1. Given 

\begin{equation}
y = f(x_1,x_2) = x_1^2 + x_1 x_2  
\end{equation}
Find its gradiant as follows [1]: 

\begin{equation}
\nabla f =  
\left(\begin{array}{c} 
\frac{ \partial f} { \partial x_1 } \\ 
\frac{ \partial f} { \partial x_2 } 
\end{array}\right)
= 
\left(\begin{array}{c} 
2 x_1 - x_2  \\ 
 - x_1 
\end{array}\right)
\end{equation}

\section{Gredient Steepest Descent for Minimization}

Now, let's define $f$ as an error function, and
we would like to minimize it. So we can try to 
change its inputs $(x_1,x_2)$ by iteration steps: 

\begin{equation}
(x_1^{k+1}, x_2^{k+1}) =
(x_1^k, x_2^k) + (- \eta \nabla f)   
\end{equation}
which will reduce the function value $f$. 
To verify this, write $f$ in terms of Taylar 
expansion as follows 

\begin{equation}
f(x_1,x_2) \simeq f(a,b) + 
\frac{ \partial f} { \partial x_1 } (x_1 - a) 
+
\frac{ \partial f} { \partial x_2 } (x_2 - b)  
\end{equation}
use simplified notation for the partial derivative
$f_{x_1}$ and $f_{x_2}$, we have 
\begin{equation}
f(x_1,x_2) \simeq f(a,b) + 
f_{x_1}(a,b) * (x_1 - a) 
+
f_{x_2}(a,b) * (x_2 - b)  
\end{equation}

we want to update $(x_1^k, x_2^k)$ to 
$(x_1^{k+1}, x_2^{k+1})$ such that
$f(x_1^{k+1}, x_2^{k+1}) < f(x_1^k, x_2^k)$. 
From equation (6), replace  
$(x_1, x_2)$ by $(x_1^{k+1}, x_2^{k+1})$, and 
let $(x_1^k, x_2^k) = (a,b)$, so we have  
 
\begin{equation}
f(x_1,x_2) - f(a,b) \simeq  
f_{x_1}(a,b) * (x_1 - a) 
+
f_{x_2}(a,b) * (x_2 - b) 
\end{equation}
Or, 
\begin{equation}
f(x_1,x_2) - f(a,b)
= (x_1 - a, x_2 - b) 
\left(\begin{array}{c} 
f_{x_1}(a,b) \\ 
f_{x_2}(a,b) \\ 
\end{array}\right)
\end{equation}
Which can be written as 
\begin{equation}
f(x_1,x_2) - f(a,b)
=(x_1 - a, x_2 - b) \nabla f 
\end{equation}

Based on the notion in Claim 1, let
\begin{equation}
\Delta x_1 = x_1 - a = -f_{x_1}  , 
\Delta x_2 = x_2 - b = -f_{x_2} 
\end{equation}
hence, we have 
\begin{equation}
f(x_1,x_2) - f(a,b)
=(\Delta x_1, \Delta x_2) \nabla f
=-(f_{x_1}^2 + f_{x_1}^2) 
\end{equation}
So appearently, 
\begin{equation}
f(x_1,x_2) - f(a,b) 
=-(f_{x_1}^2 + f_{x_1}^2) < 0  
\end{equation}
which shows the error function satisfies 
\begin{equation}
f(x_1,x_2) < f(a,b)   
\end{equation}
e.g, 
\begin{equation}
f(x_1^{k+1}, x_2^{k+1}) < f(x_1^k, x_2^k). 
\end{equation}

\section{Conclusion}
%Section 5
In this lecture note,  we describe the basic concept of 
gradient and we have noted that 
The negative gradient -$\nabla f$ follows
the direction of steepest descent. 

\section*{Acknowledgment}
The author would like to thank Kevin Lee for the 
discussion on gradient descent algorithm while on their
way back from Berkeley, California to his office.  
 
\begin{thebibliography}{1}

%\bibitem{IEEEhowto:kopka}
%H.~Kopka and P.~W. Daly, \emph{A Guide to \LaTeX}, 3rd~ed.\hskip 1em plus
%  0.5em minus 0.4em\relax Harlow, England: Addison-Wesley, 1999.

\bibitem{[1]} Gradient and partial derivatives,  
https://www.khanacademy.org/math/ multivariable-calculus/
multivariable-derivatives/ partial-derivative-and-gradient-
articles/a/the-gradient

\bibitem{[2]} Neural Networks, An Introduction, by 
B. Muller and J. Reinhardt, Springer-Verlag, 1990. 

\bibitem{B.K.P. Horn}  
\textit{B.K.P. Horn, Robot Vision}.  
MIT Press, 1982. 

\end{thebibliography}

% that's all folks
\end{document}


